\documentclass[a4paper,12pt]{article}
\usepackage[utf8]{inputenc}
\usepackage{setspace}
\usepackage{hyperref}
\usepackage{titlesec}
\usepackage{enumitem}

\titleformat{\section}{\normalfont\Large\bfseries}{\thesection}{1em}{}
\titleformat{\subsection}{\normalfont\large\bfseries}{\thesubsection}{1em}{}

\begin{document}

\title{Currículo de Ivan Moriá Borges Rodrigues}
\author{Ivan Moriá Borges Rodrigues}
\date{}
\maketitle

\begin{doublespace}

\noindent
Eu sou Ivan Moriá Borges Rodrigues, músico multi-instrumentista, musicoterapeuta com prática clínica há 7 anos, mestre em Neurociências pela UFMG e atualmente doutorando em Música pela mesma instituição.

\medskip
Ingressei no doutorado em Música no ano de 2021, pela linha de pesquisa Educação Musical, sabendo que teria que adaptar meu projeto de pesquisa - relacionado a Musicoterapia - nesta nova linha. No primeiro ano, cumpri os créditos obrigatórios e decidi migrar para a linha Sonologia, para ampliar algumas possibilidades que estava buscando na minha pesquisa. Nesta mudança, iniciei um processo de muito estudo na busca de compreender aspectos básicos de programação, amplamente utilizados na pesquisa nesta área.

\medskip
Iniciei os estudos nos softwares MAX/MSP e linguagem de programação Python. Além disso, continuei participando de congressos e outras pesquisas relacionadas à minha área inicial proposta, relacionada à Musicoterapia. Quando a linha de pesquisa em Musicoterapia especificamente foi inaugurada no PPG Música da UFMG, em 2024, entrei novamente com a solicitação ao colegiado visando mudar de linha, para, finalmente, realizar a pesquisa amparado pelo escopo teórico abordado nesta pesquisa.

\medskip
Durante minha trajetória no curso busquei participar de eventos nacionais e internacionais, com comunicações orais e pôsteres, realizando, no mínimo, uma participação em cada um deles por ano. Quando possível, sempre me voluntariei para auxiliar em algum congresso presencial, como foi o caso do 17th World Congress of Music Therapy, no qual estive também como voluntário.

\medskip
Após algum momento, comecei também a participar de bancas de graduação, avaliação de artigos em periódicos e na comissão de pesquisa da União Brasileira de Associações em Musicoterapia (UBAM), no qual organizamos o XXIV Encontro Nacional de Pesquisa em Musicoterapia, no qual também participo como avaliador dos trabalhos e moderador da mesa de abertura.

\medskip
A participação em comunicações orais em inglês em congressos (12th European Music Therapy Conference – Edinburgh, 17th World Congress of Music Therapy – Vancouver e IAMM \& ISfAM Congress, Berlin) me possibilitou uma troca de experiências muito rica e a oportunidade de explanar sobre o meu trabalho com diversos pesquisadores importantes da área. Por exemplo, após alguns dias após o congresso de Berlim, recebi um e-mail do Prof. Jörg Fachner dizendo que assistiu a minha apresentação e me convidou para avaliar um artigo em sua revista (Music \& Science- UK), relacionado à minha temática apresentada. Estas oportunidades são experiências ricas que contribuem significativamente na formação do doutorando.

\medskip
Publiquei um artigo neste período, existe um outro artigo aceito para publicação e em fase de editoração e outro artigo está sendo realizado sobre a temática Musicoterapia e Tecnologia, com uma revisão da literatura. A seguir, uma proposta de cronograma em andamento para a conclusão desta pesquisa.

\medskip
\textbf{Research Timeline}

\begin{itemize}[leftmargin=*, label={--}]
    \item (2023 – 2024): Research articles on the concept of MIR in Music Therapy, the use of MIDI data for research, and the use of technological tools in Music Therapy.
    \item Development of programming languages (2021 – 2025): such as MAX/MSP (Cycling ’74) and Python for processing MIDI data and creating visual and statistical analysis.
    \item Advancement of an intuitive visual interface (2024 – 2025): to ensure accessibility for all users and researchers.
    \item Interviews with experts in music therapy (2024 – 2025): to explore their use of technological tools and collect evaluative feedback on this ongoing research.
\end{itemize}

\medskip
Além das produções teóricas, citadas até aqui, existem também as produções artísticas, onde sempre busquei me aprimorar.

\medskip
Como violoncelista, participo da Orquestra 415 de Música Antiga, e pude me apresentar diversas vezes nos teatros do Palácio das Artes e da Biblioteca Pública, realizando um diversificado repertório barroco.

\medskip
Como acordeonista, participo da Orquestra de Choro da UFMG, onde tive várias apresentações nos concertos da Escola de Música da UFMG (Série Viva Música), e alguns concertos nos festivais municipais.

\medskip
Como bandolinista, já me apresentei como solista na Orquestra de Choro da UFMG e também no grupo independente de choro intitulado CHORODITO, onde ocasionalmente ocorrem apresentações em alguns palcos conhecidos da cidade de Belo Horizonte.

\medskip
Acesse aqui todas as produções artísticas: \url{http://ivanmoria.github.io/artisticas.html}

\end{doublespace}

\end{document}
